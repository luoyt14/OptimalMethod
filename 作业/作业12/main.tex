\documentclass[a4paper,12pt]{article}

\usepackage{fullpage} % Package to use full page
\usepackage{parskip} % Package to tweak paragraph skipping
\usepackage{tikz} % Package for drawing
\usepackage{amsmath}
\usepackage{hyperref}
\usepackage{ctex}
\usepackage{amssymb}
\usepackage{amsthm}
\usepackage{enumerate}
\usepackage{subeqnarray}
\usepackage{cases}
\renewcommand\thefigure{\thesection.\arabic{figure}}
\makeatletter
\@addtoreset{figure}{section}
\makeatother

\makeatletter
\renewcommand \theequation {%
	\ifnum \c@section>\z@ \@arabic\c@section.\fi \ifnum \c@subsection>\z@
	\@arabic\c@subsection.\fi\ifnum \c@subsubsection>\z@
	\@arabic\c@subsubsection.\fi\@arabic\c@equation}
\@addtoreset{equation}{section}
\@addtoreset{equation}{subsection}
%\setcounter{section}{-1}
\makeatother

\title{最优化方法作业12}
\author{罗雁天 \\
2018310742}
\date{\today}

\begin{document}

%\maketitle
\newcommand{\HRule}{\rule{\linewidth}{0.5mm}}
\begin{titlepage}
	\begin{center}
		% Upper part of the page
		\includegraphics[width=0.4\textwidth]{Tsinghua2.png}\\[1cm]
		\textsc{\Large \texttt{清华大学电子工程系}}\\[1cm]
		% Title
		\HRule \\[1cm]
		{\Huge \bfseries 最优化方法作业12}\\[0.4cm]
		\HRule \\[3.5cm]
		% Author and supervisor
		\begin{minipage}{0.4\textwidth}
			\begin{center}
				\Large
				\begin{tabular}{cc}
					\texttt{作者:} & 罗雁天 \\[0.5cm]
					\texttt{学号:} & 2018310742 \\[0.5cm]
					\texttt{日期:} & \today
				\end{tabular}
			\end{center}
		\end{minipage}
		\vfill
	\end{center}
\end{titlepage}

\section{}
设$p^{(1)},p^{(2)},\cdots,p^{(n)}\in \mathbb{R}^n$为一组线性无关向量,$H$是$n$阶对称正定矩阵,令向量$d^{(k)}$为:
\begin{equation}
d^{(k)}=\left\{
\begin{array}{ll}
p^{(k)},&k=1 \\
p^{(k)}-\sum_{i=1}^{k-1}\left[\frac{d^{(i)T}Hp^{(k)}}{d^{(i)T}Hd^{(i)}}\right]d^{(i)},&k=2,\cdots,n
\end{array}
\right.
\end{equation}
证明$d^{(1)},d^{(2)},\cdots,d^{(n)}$关于$H$共轭

\begin{proof}
	由数学归纳法:
\begin{itemize}
	\item 

    当$k=2$时,
    \begin{equation}
    d^{(1)T}Hd^{(2)}=p^{(1)T}H\left(p^{(1)}-\frac{p^{(1)T}Hp^{(1)}}{p^{(1)T}Hp^{(1)}}p^{(1)}\right)=0
    \end{equation}
    即$d^{(1)},d^{(2)}$关于$H$共轭;
    
    \item  假设$k<n$时,$d^{(1)},d^{(2)},\cdots,d^{(k)}$关于$H$共轭
    
    \item 那么当$k=n$时,
    \begin{equation}
    \begin{aligned}
    d^{(i)T}Hd^{(j)}&=d^{(i)T}H\left(p^{(j)}-\sum_{k=1}^{j-1}\left[\frac{d^{(k)T}Hp^{(j)}}{d^{(k)T}Hd^{(k)}}\right]d^{(k)}\right) \\
    &=d^{(i)T}Hp^{(j)}-\sum_{k=1}^{j-1}\left[\frac{d^{(k)T}Hp^{(j)}}{d^{(k)T}Hd^{(k)}}\right]d^{(i)T}Hd^{(k)} \\
    &=d^{(i)T}Hp^{(j)}-\left[\frac{d^{(i)T}Hp^{(j)}}{d^{(i)T}Hd^{(i)}}\right]d^{(i)T}Hd^{(i)} \\
    &= d^{(i)T}Hp^{(j)}-d^{(i)T}Hp^{(j)} \\
    &=0
    \end{aligned}
    \end{equation}
    即,$d^{(i)},d^{(j)}$关于$H$共轭
\end{itemize}
    综上所述,由数学归纳法,$d^{(1)},d^{(2)},\cdots,d^{(n)}$关于$H$共轭
\end{proof}

\newpage
\section{}
设将FR共轭梯度法用于有三个变量的函数$f(x)$,第1次迭代,搜索方向$d^{(1)}=[1,-1,2]^T$,沿$d^{(1)}$作精确一维搜索,得到点$x^{(2)}$,又设
\begin{equation}
\frac{\partial f(x^{(2)})}{\partial x_1}=-2,\quad \frac{\partial f(x^{(2)})}{\partial x_2}=-2
\end{equation}
那么按照共轭梯度法的规定,从$x^{(2)}$出发的搜索方向是什么?

\begin{proof}[解]
	记$g_i=\bigtriangledown f(x^{(i)})$.由一维搜索可知,$g_2^Td^{(1)}=0$,解得:$g_2=[-2,-2,0]^T$
	
	根据FR共轭梯度法规定:
	\begin{equation}
	\begin{aligned}
	&g_1=-d^{(1)}=[-1,1,-2]^T \\
	&\beta_1=\frac{||g_2||^2}{||g_1||^2}=\frac{4}{3}
	\end{aligned}
	\end{equation}
	
	所以,$d^{(2)}=-g_2+\beta_1 d^{(1)}=[\frac{10}{3},\frac{2}{3},\frac{8}{3}]^T$
\end{proof}
\end{document}