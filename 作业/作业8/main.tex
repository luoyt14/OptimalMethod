\documentclass[a4paper]{article}

\usepackage{fullpage} % Package to use full page
\usepackage{parskip} % Package to tweak paragraph skipping
\usepackage{tikz} % Package for drawing
\usepackage{amsmath}
\usepackage{hyperref}
\usepackage{ctex}
\usepackage{amssymb}
\usepackage{amsthm}

\renewcommand\thefigure{\thesection.\arabic{figure}}
\makeatletter
\@addtoreset{figure}{section}
\makeatother

\makeatletter
\renewcommand \theequation {%
	\ifnum \c@section>\z@ \@arabic\c@section.\fi \ifnum \c@subsection>\z@
	\@arabic\c@subsection.\fi\ifnum \c@subsubsection>\z@
	\@arabic\c@subsubsection.\fi\@arabic\c@equation}
\@addtoreset{equation}{section}
\@addtoreset{equation}{subsection}
%\setcounter{section}{-1}
\makeatother

\title{最优化方法作业8}
\author{罗雁天 \\
2018310742}
\date{\today}

\begin{document}

%\maketitle
\newcommand{\HRule}{\rule{\linewidth}{0.5mm}}
\begin{titlepage}
	\begin{center}
		% Upper part of the page
		\includegraphics[width=0.4\textwidth]{Tsinghua2.png}\\[1cm]
		\textsc{\Large \texttt{清华大学电子工程系}}\\[1cm]
		% Title
		\HRule \\[1cm]
		{\Huge \bfseries 最优化方法作业8}\\[0.4cm]
		\HRule \\[3.5cm]
		% Author and supervisor
		\begin{minipage}{0.4\textwidth}
			\begin{center}
				\Large
				\begin{tabular}{cc}
					\texttt{作者:} & 罗雁天 \\[0.5cm]
					\texttt{学号:} & 2018310742 \\[0.5cm]
					\texttt{日期:} & \today
				\end{tabular}
			\end{center}
		\end{minipage}
		\vfill
	\end{center}
\end{titlepage}

\section{}
给定非线性规划问题:
\begin{equation}
\begin{aligned}
\max \quad &b^Tx \qquad x\in \mathbb{R}^n \\
s.t.\quad &x^Tx\le 1
\end{aligned}
\end{equation}
其中$b\ne 0$,证明向量$\bar{x}=\frac{b}{||b||}$满足最优性的充分条件

\begin{proof}
	原问题可以化为如下的优化问题:
	\begin{equation}
	\label{eq1}
	\begin{aligned}
	\min \quad &-b^Tx \qquad x\in \mathbb{R}^n\\
	s.t. \quad &-x^Tx+1\ge 0
	\end{aligned}
	\end{equation}
	KKT条件如下:
	\begin{equation}
	\begin{aligned}
	-b+2wx&=0\\
	w(-x^Tx+1)&=0 \\
	w&\ge 0
	\end{aligned}
	\end{equation}
	可以验证$\bar{x}=\frac{b}{||b||}$是KKT点,此时$w=\frac{1}{2}||b||>0$符合条件。又由于目标函数是线性函数,可以看成凸函数,$\bigtriangledown^2(g)=-2$,所以约束条件是凹函数,因此规划问题\ref{eq1}为凸规划问题。因此,向量$\bar{x}=\frac{b}{||b||}$满足最优性的充分条件。
\end{proof}

\section{}
给定非线性规划问题:
\begin{equation}
\begin{aligned}
\min \quad &c^Tx \\
s.t. \quad &Ax=0 \\
& x^Tx \le \gamma^2
\end{aligned}
\end{equation}
其中$A$为$m\times n$的矩阵($m<n$),$A$的秩为$m$, $c\in \mathbb{R}^n$且$c\neq 0$, $\gamma$是一个正数,试求问题的最优解及目标函数最优值。

\textbf{解:}

原问题可以化为如下规划问题:

\begin{equation}
\begin{aligned}
\min \quad &c^Tx \\
s.t. \quad &Ax=0 \\
& -x^Tx + \gamma^2 \ge 0
\end{aligned}
\end{equation}

Lagrange函数为:
\begin{equation}
L(x,w,v)=c^Tx-w(-x^Tx+\gamma^2)-v^T(Ax)
\end{equation}
KKT条件如下:

\begin{equation}
\label{eq2}
\left\{
\begin{array}{c}
c-A^Tv+2wx=0 \\
w(x^Tx-\gamma^2)=0 \\
x^Tx-\gamma^2\ge 0 \\
Ax=0 \\
w\ge 0
\end{array}
\right.
\end{equation}
由于$A$行满秩,因此$AA^T$可逆,解式\ref{eq2}得:

\begin{equation}
\begin{aligned}
v&=(AA^T)^{-1}Ac \\
w&=-\frac{f_{min}}{2\gamma^2} \\
f_{min}&=-\gamma\sqrt{c^T(c-A^Tv)} \\
x&=\frac{\gamma^2}{f_{min}}(c-A^Tv)\quad(f_{min}\ne 0)
\end{aligned}
\end{equation}

又由于原问题为凸规划问题,所以:
\begin{enumerate}
	\item 当$c=A^Tv$时,最优解不唯一,最优值为$f_{min}=0$
	\item 当$c\neq A^Tv$时,最优值为$f_{min}=-\gamma\sqrt{c^T(c-A^Tv)}$,最优解为$x=\frac{\gamma^2}{f_{min}}(c-A^Tv)\quad(f_{min}\ne 0)$
\end{enumerate}
\end{document}