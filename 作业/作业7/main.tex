\documentclass[a4paper]{article}

\usepackage{fullpage} % Package to use full page
\usepackage{parskip} % Package to tweak paragraph skipping
\usepackage{tikz} % Package for drawing
\usepackage{amsmath}
\usepackage{hyperref}
\usepackage{ctex}
\usepackage{amssymb}
\usepackage{amsthm}

\renewcommand\thefigure{\thesection.\arabic{figure}}
\makeatletter
\@addtoreset{figure}{section}
\makeatother

\makeatletter
\renewcommand \theequation {%
	\ifnum \c@section>\z@ \@arabic\c@section.\fi \ifnum \c@subsection>\z@
	\@arabic\c@subsection.\fi\ifnum \c@subsubsection>\z@
	\@arabic\c@subsubsection.\fi\@arabic\c@equation}
\@addtoreset{equation}{section}
\@addtoreset{equation}{subsection}
%\setcounter{section}{-1}
\makeatother

\title{最优化方法作业7}
\author{罗雁天 \\
2018310742}
\date{\today}

\begin{document}

%\maketitle
\newcommand{\HRule}{\rule{\linewidth}{0.5mm}}
\begin{titlepage}
	\begin{center}
		% Upper part of the page
		\includegraphics[width=0.4\textwidth]{Tsinghua2.png}\\[1cm]
		\textsc{\Large \texttt{清华大学电子工程系}}\\[1cm]
		% Title
		\HRule \\[1cm]
		{\Huge \bfseries 最优化方法作业7}\\[0.4cm]
		\HRule \\[3.5cm]
		% Author and supervisor
		\begin{minipage}{0.4\textwidth}
			\begin{center}
				\Large
				\begin{tabular}{cc}
					\texttt{作者:} & 罗雁天 \\[0.5cm]
					\texttt{学号:} & 2018310742 \\[0.5cm]
					\texttt{日期:} & \today
				\end{tabular}
			\end{center}
		\end{minipage}
		\vfill
	\end{center}
\end{titlepage}

\section{}
$f(x_1,x_2)=10-2(x_2-x_1^2)^2,S=\{(x_1,x_2)|-11\le x_1 \le 1, -1\le x_2\le 1\}$,判断$f(x_1,x_2)$是否为$S$上的凸函数?

\textbf{解}:

计算$f$的梯度和Hessian矩阵如下:
\begin{equation}
\begin{aligned}
\bigtriangledown f &=\left[\frac{\partial f}{\partial x_1},\frac{\partial f}{\partial x_2}\right]^T=[8x_1(x_2-x_1^2),-4(x_2-x_1^2)]^T \\
\bigtriangledown^2 f &= 
\left[
\begin{array}{cc}
\frac{\partial^2 f}{\partial x_1^2} & \frac{\partial^2  f}{\partial x_1\partial x_2} \\
\frac{\partial^2 f}{\partial x_2\partial x_1} & \frac{\partial^2 f}{\partial x_2^2}
\end{array}
\right]=
\left[
\begin{array}{cc}
8x_2-24x_1^2 & 8x_1 \\
8x_1 & -4
\end{array}
\right]
\end{aligned}
\end{equation}

由于$x=[0,1]^T\in S$,但是$\bigtriangledown^2 f(x)=\left[
\begin{array}{cc}
8 & 0 \\
0 & -4
\end{array}
\right]$非半正定矩阵,因此$f(x_1,x_2)$不是$S$上的凸函数

\section{}
给定函数
\begin{equation*}
f(x)=\frac{x_1+x_2}{3+x_1^2+x_2^2+x_1x_2}
\end{equation*}
求$f(x)$的极小值点

\textbf{解:}

计算$\bigtriangledown f(x)$和$\bigtriangledown^2 f(x)$如下:
\begin{equation}
\begin{aligned}
\bigtriangledown f(x) &=\left[\frac{3-x_1^2-2x_1x_2}{(3+x_1^2+x_2^2+x_1x_2)^2},\frac{3-x_2^2-2x_1x_2}{(3+x_1^2+x_2^2+x_1x_2)^2}\right]^T \\
\bigtriangledown^2 f(x) &= 
\left[
\begin{array}{cc}
\frac{-18x_1-12x_2+2x_1^3-2x_2^3+6x_1^2x_2}{(3+x_1^2+x_2^2+x_1x_2)^3} & \frac{-12x_1-12x_2+6x_1^2x_2+6x_1x_2^2}{(3+x_1^2+x_2^2+x_1x_2)^3} \\
\frac{-12x_1-12x_2+6x_1^2x_2+6x_1x_2^2}{(3+x_1^2+x_2^2+x_1x_2)^3}  & \frac{-12x_1-18x_2-2x_1^3+2x_2^3+6x_1x_2^2}{(3+x_1^2+x_2^2+x_1x_2)^3}
\end{array}
\right]
\end{aligned}
\end{equation}

令$\bigtriangledown f(x)=0$得$\textbf{x}^{(1)}=[1,1]^T,\textbf{x}^{(2)}=[-1,-1]^T$,代入$\bigtriangledown^2 f(x)$中得:
\begin{equation}
\begin{aligned}
\bigtriangledown^2 f(\textbf{x}^{(1)}) &= 
\left[
\begin{array}{cc}
-\frac{1}{9} & -\frac{1}{18} \\
-\frac{1}{18}  & -\frac{1}{9}
\end{array}
\right] \\
\bigtriangledown^2 f(\textbf{x}^{(2)}) &= 
\left[
\begin{array}{cc}
\frac{1}{9} & \frac{1}{18} \\
\frac{1}{18}  & \frac{1}{9}
\end{array}
\right]
\end{aligned}
\end{equation}
由于$\bigtriangledown^2 f(\textbf{x}^{(1)})$负定,$\bigtriangledown^2 f(\textbf{x}^{(2)})$正定,因此$\textbf{x}^{(2)}=[-1,-1]^T$为极小点

\section{}
给定非线性规划问题:
\begin{equation*}
\begin{aligned}
\min\quad &f(x_1,x_2)=(x_1-\frac{9}{4})^2+(x_2-2)^2 \\
s.t.\quad &-x_1^2+x_2\ge 0 \\
&x_1+x_2\le 6 \\
&x_1,x_2\ge 0
\end{aligned}
\end{equation*}
判断下列各点是否为最优解:$x^{(1)}=\left[\frac{3}{2},\frac{9}{4}\right]^T,x^{(2)}=\left[\frac{9}{4},2\right]^T,x^{(3)}=[0,2]^T$

\textbf{解:}

原规划问题可以化为如下凸规划问题:
\begin{equation*}
\begin{aligned}
\min\quad &f(x_1,x_2)=(x_1-\frac{9}{4})^2+(x_2-2)^2 \\
s.t.\quad &-x_1^2+x_2\ge 0 \\
&-x_1-x_2+6 \ge 0 \\
&x_1,x_2 \ge 0
\end{aligned}
\end{equation*}

因此我们只需要验证KKT条件即可。计算梯度如下,
\begin{equation}
\begin{aligned}
\bigtriangledown f(x) &=\left[2(x_1-\frac{9}{4}),2(x_2-2)\right]^T \\
\bigtriangledown g_1(x)&=[-2x_1,1]^T \\
\bigtriangledown g_2(x)&=[-1,-1]^T \\
\bigtriangledown g_3(x)&=[1,0]^T \\
\bigtriangledown g_4(x)&=[0,1]^T \\ 
\end{aligned}
\end{equation}

\begin{itemize}
	
\item 对$x^{(1)}=\left[\frac{3}{2},\frac{9}{4}\right]^T$,
\begin{equation}
\left\{
\begin{array}{c}
-\frac{3}{2}+3w_1=0 \\
\frac{1}{2}-w_1=0 \\
w_1 \ge 0
\end{array}
\right.
\end{equation}
解得$w_1=\frac{1}{2}$,满足KKT条件,因此$x^{(1)}=\left[\frac{3}{2},\frac{9}{4}\right]^T$是最优解,最优值为$\frac{5}{8}$。

\item 对$x^{(2)}=\left[\frac{9}{4},2\right]^T$,由于不满足第一个不等式约束,因此不是可行解。

\item 对$x^{(3)}=[0,2]^T$,
\begin{equation}
\left\{
\begin{array}{c}
-\frac{9}{2}-w_3=0 \\
0=0 \\
w_3 \ge 0
\end{array}
\right.
\end{equation}
由第一个式子解得$w_3=-\frac{9}{2}$,不满足第三个式子,因此$x^{(3)}=[0,2]^T$不是最优解

\end{itemize}

\section{}
求原点$x^{0}=[0,0]^T$到凸集$S=\{x|x_1+x_2\ge4,2x_1+x_2\ge5\}$的最小距离

\textbf{解:}

原问题即是求解如下规划问题:
\begin{equation*}
\begin{aligned}
\min\quad &f(x_1,x_2)=x_1^2+x_2^2 \\
s.t.\quad &x_1+x_2-4\ge 0 \\
&2x_1+x_2-5 \ge 0
\end{aligned}
\end{equation*}

计算梯度如下,
\begin{equation}
\begin{aligned}
\bigtriangledown f(x) &=[2x_1,2x_2]^T \\
\bigtriangledown g_1(x)&=[1,1]^T \\
\bigtriangledown g_2(x)&=[2,1]^T
\end{aligned}
\end{equation}

KKT条件如下:
\begin{equation}
\left\{
\begin{array}{c}
2x_1-w_1-2w_2=0 \\
2x_2-w_1-w_2=0 \\
w_1(x_1+x_2-4)=0 \\
w_2(2x_1+x_2-5)=0 \\
w_1\ge 0 \\
w_2\ge 0
\end{array}
\right.
\end{equation}
解得$x_1=2,x_2=2,w_1=4,w_2=0$满足KKT条件,因此最优解为$\bar{x}=[2,2]^T$,最小距离为$2\sqrt{2}$
%计算$\bigtriangledown f(x)$和$\bigtriangledown^2 f(x)$如下:
%\begin{equation}
%\begin{aligned}
%\bigtriangledown f(x) &=\left[2(x_1-\frac{9}{4}),2(x_2-2)\right]^T \\
%\bigtriangledown^2 f(x) &= 
%\left[
%\begin{array}{cc}
%2 & 0 \\
%0  & 2
%\end{array}
%\right]
%\end{aligned}
%\end{equation}


\end{document}