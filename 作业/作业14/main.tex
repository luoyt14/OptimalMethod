\documentclass[a4paper,12pt]{article}

\usepackage{fullpage} % Package to use full page
\usepackage{parskip} % Package to tweak paragraph skipping
\usepackage{tikz} % Package for drawing
\usepackage{amsmath}
\usepackage{hyperref}
\usepackage{ctex}
\usepackage{amssymb}
\usepackage{amsthm}
\usepackage{enumerate}
\usepackage{subeqnarray}
\usepackage{cases}
\renewcommand\thefigure{\thesection.\arabic{figure}}
\makeatletter
\@addtoreset{figure}{section}
\makeatother

\makeatletter
\renewcommand \theequation {%
	\ifnum \c@section>\z@ \@arabic\c@section.\fi \ifnum \c@subsection>\z@
	\@arabic\c@subsection.\fi\ifnum \c@subsubsection>\z@
	\@arabic\c@subsubsection.\fi\@arabic\c@equation}
\@addtoreset{equation}{section}
\@addtoreset{equation}{subsection}
%\setcounter{section}{-1}
\makeatother

\title{最优化方法作业14}
\author{罗雁天 \\
2018310742}
\date{\today}

\begin{document}

%\maketitle
\newcommand{\HRule}{\rule{\linewidth}{0.5mm}}
\begin{titlepage}
	\begin{center}
		% Upper part of the page
		\includegraphics[width=0.4\textwidth]{Tsinghua2.png}\\[1cm]
		\textsc{\Large \texttt{清华大学电子工程系}}\\[1cm]
		% Title
		\HRule \\[1cm]
		{\Huge \bfseries 最优化方法作业14}\\[0.4cm]
		\HRule \\[3.5cm]
		% Author and supervisor
		\begin{minipage}{0.4\textwidth}
			\begin{center}
				\Large
				\begin{tabular}{cc}
					\texttt{作者:} & 罗雁天 \\[0.5cm]
					\texttt{学号:} & 2018310742 \\[0.5cm]
					\texttt{日期:} & \today
				\end{tabular}
			\end{center}
		\end{minipage}
		\vfill
	\end{center}
\end{titlepage}

\section{}
考虑下列问题:
\begin{equation}
\begin{aligned}
\min \quad &x_1x_2 \\
s.t. \quad &g(x)=-2x_1+x_2+3\ge 0
\end{aligned}
\end{equation}
\begin{enumerate}[(1)]
	\item 用二阶最优性条件证明点$\bar{x}=\left[\frac{3}{4},-\frac{3}{2}\right]^T$是局部最优解,并说明它是否为全局最优解
	\item 定义障碍函数为$G(x,r)=x_1x_2-r\ln g(x)$,试用内点法求解此问题,并说明内点法产生的序列趋向点$\bar{x}$
\end{enumerate}

\begin{proof}[解]
	\begin{enumerate}[(1)]
		\item 计算梯度如下:
		\begin{equation}
		\begin{aligned}
		\bigtriangledown f(\bar{x})&=\left[
		\begin{array}{c}
		x_2 \\
		x_1
		\end{array}
		\right]=\left[
		\begin{array}{c}
		-\frac{3}{2} \\
		\frac{3}{4}
		\end{array}
		\right]\\
		\bigtriangledown g(\bar{x})&=\left[
		\begin{array}{c}
		-2 \\
		1
		\end{array}
		\right]
		\end{aligned}
		\end{equation}
		由于$g(x)\ge 0$是起作用约束,因此令$\bigtriangledown f(\bar{x})-w\bigtriangledown g(\bar{x})=0$得$w=\frac{3}{4}>0$,因此$\bar{x}$是KKT点
		
		取Lagrange函数为$L(x,w)=x_1x_2-w(-2x_1+x_2+3)$,计算其Hessian矩阵如下:
		\begin{equation}
		\bigtriangledown^2_x L(x,w)=\left[
		\begin{array}{cc}
		0 & 1 \\
		1 & 0
		\end{array}
		\right]
		\end{equation}
		令:
		\begin{equation}
		\bigtriangledown g(\bar{x})^T\mathbf{d}=[-2,1]\left[
		\begin{array}{c}
		d_1 \\
		d_2
		\end{array}
		\right]=0
		\end{equation}
		得$d_2=2d_1$,因此方向集为:
		\begin{equation}
		G=\{\mathbf{d}|d_2=2d_1,\mathbf{d}\ne 0\}
		\end{equation}
		因此对$\forall \mathbf{d}\in G$有:
		\begin{equation}
		\mathbf{d}^T\bigtriangledown^2_x L(x,w)\mathbf{d}=4d_1^2>0
		\end{equation}
		因此$\bar{x}=\left[\frac{3}{4},-\frac{3}{2}\right]^T$是局部最优解。取$x=[-3,3]^T$目标函数值为$-9$比$\bar{x}$处的目标函数值更小,因此$\bar{x}$不是全局最优解
		
		\item 令:
		\begin{equation}
		\left\{
		\begin{array}{c}
		\frac{\partial G(x,r)}{\partial x_1} =x_2+\frac{2r}{-2x_1+x_2+3}=0 \\
		\frac{\partial G(x,r)}{\partial x_2} =x_1-\frac{r}{-2x_2+x_2+3}=0
		\end{array}
		\right.
		\end{equation}
		得:
		\begin{equation}
		x(r)=\left(\frac{3+\sqrt{9-16r}}{8},-\frac{3+\sqrt{9-16r}}{4}\right)^T
		\end{equation}
		令$r\to 0$得:
		\begin{equation}
		x(r)\to \left(\frac{3}{4},-\frac{3}{2}\right)^T=\bar{x}
		\end{equation}
	\end{enumerate}
\end{proof}

\end{document}
