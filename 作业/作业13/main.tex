\documentclass[a4paper,12pt]{article}

\usepackage{fullpage} % Package to use full page
\usepackage{parskip} % Package to tweak paragraph skipping
\usepackage{tikz} % Package for drawing
\usepackage{amsmath}
\usepackage{hyperref}
\usepackage{ctex}
\usepackage{amssymb}
\usepackage{amsthm}
\usepackage{enumerate}
\usepackage{subeqnarray}
\usepackage{cases}
\renewcommand\thefigure{\thesection.\arabic{figure}}
\makeatletter
\@addtoreset{figure}{section}
\makeatother

\makeatletter
\renewcommand \theequation {%
	\ifnum \c@section>\z@ \@arabic\c@section.\fi \ifnum \c@subsection>\z@
	\@arabic\c@subsection.\fi\ifnum \c@subsubsection>\z@
	\@arabic\c@subsubsection.\fi\@arabic\c@equation}
\@addtoreset{equation}{section}
\@addtoreset{equation}{subsection}
%\setcounter{section}{-1}
\makeatother

\title{最优化方法作业13}
\author{罗雁天 \\
2018310742}
\date{\today}

\begin{document}

%\maketitle
\newcommand{\HRule}{\rule{\linewidth}{0.5mm}}
\begin{titlepage}
	\begin{center}
		% Upper part of the page
		\includegraphics[width=0.4\textwidth]{Tsinghua2.png}\\[1cm]
		\textsc{\Large \texttt{清华大学电子工程系}}\\[1cm]
		% Title
		\HRule \\[1cm]
		{\Huge \bfseries 最优化方法作业13}\\[0.4cm]
		\HRule \\[3.5cm]
		% Author and supervisor
		\begin{minipage}{0.4\textwidth}
			\begin{center}
				\Large
				\begin{tabular}{cc}
					\texttt{作者:} & 罗雁天 \\[0.5cm]
					\texttt{学号:} & 2018310742 \\[0.5cm]
					\texttt{日期:} & \today
				\end{tabular}
			\end{center}
		\end{minipage}
		\vfill
	\end{center}
\end{titlepage}

\section{}
考虑下列问题:
\begin{equation}
\begin{aligned}
\min\quad &x_1^2+x_1x_2+2x_2^2-6x_1-2x_2-12x_3 \\
s.t.\quad &x_1+x_2+x_3=2 \\
&-x_1+2x_2\le 3\\
&x_1,x_2,x_3\ge 0
\end{aligned}
\end{equation}
求出在点$\hat{x}=(1,1,0)^T$处的一个下降可行方向.
\begin{proof}[解]
目标函数$f(x)=x_1^2+x_1x_2+2x_2^2-6x_1-2x_2-12x_3$,计算$\hat{x}$处导数如下:
\begin{equation}
\bigtriangledown f(\hat{x})=\left[
\begin{array}{c}
2x_1+x_2-6 \\
x_1+4x_2-2 \\
-12
\end{array}
\right]=\left[
\begin{array}{c}
-3 \\
3 \\
-12
\end{array}
\right]
\end{equation}

在$\hat{x}=[1,1,0]$处的起作用约束有:

\begin{equation}
\begin{aligned}
x_1+x_2+x_3&=2 \\
x_3 \ge 0
\end{aligned}
\end{equation}

因此在$\hat{x}$处的可行方向满足以下条件:
\begin{equation}
\left\{
\begin{aligned}
d_1+d_2+d_3&=0 \\
d_3&\ge 0 \\
\bigtriangledown f(\hat{x})^Td<0 &\Rightarrow -3d_1+3d_2-12d_3<0
\end{aligned}
\right.
\end{equation}

因此可以取$d=[0,-1,1]^T$作为下降可行方向
\end{proof}

\section{}
考虑下列问题:
\begin{equation}
\begin{aligned}
\min\quad &f(x) \\
s.t.\quad &g_i(x)\ge 0\quad i=1,2,\cdots,m \\
&h_j(x)=0\quad j=1,2,\cdots,l
\end{aligned}
\end{equation}
设$\hat{x}$是可行点,$I=\{i|g_i(\hat{x})=0\}$

证明$\hat{x}$为KKT点的充要条件是下列问题的目标函数的最优值为零:
\begin{equation}
\label{LP}
\begin{aligned}
\min \quad& \bigtriangledown f(\hat{x})^Td\\
s.t. \quad& \bigtriangledown g_i(\hat{x})^Td\ge 0\quad i\in I \\
& \bigtriangledown h_j(\hat{x})^Td=0\quad j=1,2,\cdots,l \\
& -1\le d_j\le 1\quad j=1,2,\cdots,n
\end{aligned}
\end{equation}

\begin{proof}
	$\hat{x}$为KKT点的充要条件为,存在乘子$w_i\ge 0(i\in I)$和$v_j(j=1,2,\cdots,l)$使得:
	\begin{equation}
	\label{1}
	\bigtriangledown f(\hat{x})-\sum_{i\in I}w_i\bigtriangledown g_i(\hat{x})-\sum_{j=1}^{l}v_j\bigtriangledown h_j(\hat{x})=0
	\end{equation}
	
	设$A_1=[\bigtriangledown g_{i_1}(\hat{x}),\bigtriangledown g_{i_2}(\hat{x}),\cdots,\bigtriangledown g_{i_k}(\hat{x})]$,$w=[w_1,w_2,\cdots,w_k]^T$,
	
	$B=[\bigtriangledown h_1(\hat{x}),\bigtriangledown h_2(\hat{x}),\cdots,\bigtriangledown h_l(\hat{x}),]$,$v=[v_1,v_2,\cdots,v_l]^T=p-q$
	
	则式(\ref{1})可以写作:
	
	\begin{equation}
	\label{2}
	\left[-A_1,-B,B\right]\left[
	\begin{array}{c}
	w\\p\\q
	\end{array}\right]=-\bigtriangledown f(\hat{x}),\qquad
	\left[
	\begin{array}{c}
	w\\p\\q
	\end{array}\right]\ge 0
	\end{equation}
	
	根据Farkas定理,系统(\ref{2})有解得充要条件是系统(\ref{3})无解。
	
	\begin{equation}
	\label{3}
	\left[
	\begin{array}{c}
	-A_1^T\\-B^T\\B^T
	\end{array}
	\right]d\ge 0,\qquad -\bigtriangledown f(\hat{x})d>0
	\end{equation}
	
	即:
	\begin{equation}
	\left\{
	\begin{array}{c}
	\bigtriangledown f(\hat{x})d<0 \\
	A_1^T\ge 0 \\
	B^Td=0
	\end{array}
	\right.
	\end{equation}
	无解。因此线性规划(\ref{LP})的最优值为0
\end{proof}
\end{document}
