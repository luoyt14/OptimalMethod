\documentclass[a4paper]{article}

\usepackage{fullpage} % Package to use full page
\usepackage{parskip} % Package to tweak paragraph skipping
\usepackage{tikz} % Package for drawing
\usepackage{amsmath}
\usepackage{hyperref}
\usepackage{ctex}
\usepackage{amssymb}
\usepackage{amsthm}

\renewcommand\thefigure{\thesection.\arabic{figure}}
\makeatletter
\@addtoreset{figure}{section}
\makeatother

\makeatletter
\renewcommand \theequation {%
	\ifnum \c@section>\z@ \@arabic\c@section.\fi \ifnum \c@subsection>\z@
	\@arabic\c@subsection.\fi\ifnum \c@subsubsection>\z@
	\@arabic\c@subsubsection.\fi\@arabic\c@equation}
\@addtoreset{equation}{section}
\@addtoreset{equation}{subsection}
%\setcounter{section}{-1}
\makeatother

\title{最优化方法作业10}
\author{罗雁天 \\
2018310742}
\date{\today}

\begin{document}

%\maketitle
\newcommand{\HRule}{\rule{\linewidth}{0.5mm}}
\begin{titlepage}
	\begin{center}
		% Upper part of the page
		\includegraphics[width=0.4\textwidth]{Tsinghua2.png}\\[1cm]
		\textsc{\Large \texttt{清华大学电子工程系}}\\[1cm]
		% Title
		\HRule \\[1cm]
		{\Huge \bfseries 最优化方法作业10}\\[0.4cm]
		\HRule \\[3.5cm]
		% Author and supervisor
		\begin{minipage}{0.4\textwidth}
			\begin{center}
				\Large
				\begin{tabular}{cc}
					\texttt{作者:} & 罗雁天 \\[0.5cm]
					\texttt{学号:} & 2018310742 \\[0.5cm]
					\texttt{日期:} & \today
				\end{tabular}
			\end{center}
		\end{minipage}
		\vfill
	\end{center}
\end{titlepage}

\section{}
定义算法映射如下:
\begin{equation}
A(x)=\left\{
\begin{array}{cc}
\left[\frac{3}{2}+\frac{1}{4}x,1+\frac{1}{2}x\right] & \mbox{当}x\ge 2 \\
\frac{1}{2}(x+1) & \mbox{当}x<2
\end{array}
\right.
\end{equation}
证明$A$在$x=2$处不是闭的

\begin{proof}
	取$x^{(k)}=2-\frac{1}{k}$,则$y^{(k)}=\frac{1}{2}(2-\frac{1}{k}+1)=\frac{3}{2}-\frac{1}{2k}$
	
	当$k\to +\infty$时,$\bar{x}=\lim\limits_{k\to +\infty}x^{(k)}=2,\bar{y}=\lim\limits_{k\to +\infty}y^{(k)}=\frac{3}{2}$
	
	但$A(\bar{x})=2\neq \bar{y}$,因此,$A$在$x=2$处不是闭的
\end{proof}

\section{}
在集合$X=[0,1]$上定义算法映射:
\begin{equation}
A(x)=\left\{
\begin{array}{cc}
[0,x) & 0<x\le 1 \\
0 & x=0
\end{array}
\right.
\end{equation}
讨论在以下各点处$A$是否为闭的: $x^{(1)}=0,x^{(2)}=\frac{1}{2}$
\begin{proof}[解]
	
	
	\begin{itemize}
		\item 对于$x^{(1)}=0$,$A$在该点处是闭的;
		\item 对于$x^{(2)}=\frac{1}{2}$,设$x^{(k)}=\frac{1}{2}+\frac{1}{k}, y^{(k)}=\frac{1}{2}+\frac{1}{2k}$
		
		所以$\bar{x}=\lim\limits_{k\to +\infty}x^{(k)}=\frac{1}{2},\bar{y}=\lim\limits_{k\to +\infty}y^{(k)}=\frac{1}{2}$
		
		由于$A(\bar{x})=[0,\frac{1}{2})$,因此$\bar{y}=\frac{1}{2}\notin A(\bar{x})$,所以$A$在$x^{(2)}=\frac{1}{2}$处不是闭的
	\end{itemize}

\end{proof}


\end{document}