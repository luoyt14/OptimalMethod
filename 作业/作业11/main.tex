\documentclass[a4paper]{article}

\usepackage{fullpage} % Package to use full page
\usepackage{parskip} % Package to tweak paragraph skipping
\usepackage{tikz} % Package for drawing
\usepackage{amsmath}
\usepackage{hyperref}
\usepackage{ctex}
\usepackage{amssymb}
\usepackage{amsthm}
\usepackage{enumerate}
\usepackage{subeqnarray}
\usepackage{cases}
\renewcommand\thefigure{\thesection.\arabic{figure}}
\makeatletter
\@addtoreset{figure}{section}
\makeatother

\makeatletter
\renewcommand \theequation {%
	\ifnum \c@section>\z@ \@arabic\c@section.\fi \ifnum \c@subsection>\z@
	\@arabic\c@subsection.\fi\ifnum \c@subsubsection>\z@
	\@arabic\c@subsubsection.\fi\@arabic\c@equation}
\@addtoreset{equation}{section}
\@addtoreset{equation}{subsection}
%\setcounter{section}{-1}
\makeatother

\title{最优化方法作业11}
\author{罗雁天 \\
2018310742}
\date{\today}

\begin{document}

%\maketitle
\newcommand{\HRule}{\rule{\linewidth}{0.5mm}}
\begin{titlepage}
	\begin{center}
		% Upper part of the page
		\includegraphics[width=0.4\textwidth]{Tsinghua2.png}\\[1cm]
		\textsc{\Large \texttt{清华大学电子工程系}}\\[1cm]
		% Title
		\HRule \\[1cm]
		{\Huge \bfseries 最优化方法作业11}\\[0.4cm]
		\HRule \\[3.5cm]
		% Author and supervisor
		\begin{minipage}{0.4\textwidth}
			\begin{center}
				\Large
				\begin{tabular}{cc}
					\texttt{作者:} & 罗雁天 \\[0.5cm]
					\texttt{学号:} & 2018310742 \\[0.5cm]
					\texttt{日期:} & \today
				\end{tabular}
			\end{center}
		\end{minipage}
		\vfill
	\end{center}
\end{titlepage}

\section{}
给定函数$f(x)=(6+x_1+x_2)^2+(2-3x_1-3x_2-x_1x_2)^2$,求在点$\hat{x}=[-4,6]^T$处的牛顿方向和最速下降方向

\begin{proof}[解]
	计算$\bigtriangledown f(x)$和$\bigtriangledown^2 f(x)$如下:
	\begin{equation}
	\begin{aligned}
	\bigtriangledown f(x)&=\left[
	\begin{array}{c}
	2(6+x_1+x_2)+2(2-3x_1-3x_2-x_1x_2)(-3-x_2) \\
	2(6+x_1+x_2)+2(2-3x_1-3x_2-x_1x_2)(-3-x_1)
	\end{array}
	\right] \\
	\bigtriangledown^2 f(x)&=\left[
	\begin{array}{cc}
	2(x_2^2+6x_2+10) & 2(2x_1x_2+6(x_1+x_2)+8) \\
	2(2x_1x_2+6(x_1+x_2)+8) & 2(x_1^2+6x_1+10)
	\end{array}
	\right]
	\end{aligned}
	\end{equation}
	所以,$\bigtriangledown f(\hat{x})$和$\bigtriangledown^2 f(\hat{x})$为:
	\begin{equation}
	\begin{aligned}
	\bigtriangledown f(\hat{x})&=\left[
	\begin{array}{c}
	-344 \\
	56
	\end{array}
	\right] \\
	\bigtriangledown^2 f(\hat{x})&=\left[
	\begin{array}{cc}
	164 & -56 \\
	-56 & 4
	\end{array}
	\right] \\
	\bigtriangledown^2 f(\hat{x})^{-1}&=-\frac{1}{2480}\left[
	\begin{array}{cc}
	4 & 56 \\
	56 & 164
	\end{array}
	\right]
	\end{aligned} 
	\end{equation}
	
	\begin{enumerate}[(i)]
		\item $\hat{x}$处的牛顿方向为$d=-\bigtriangledown^2f(\hat{x})^{-1}\bigtriangledown f(\hat{x})=\left[\frac{22}{31},-\frac{126}{31}\right]^T$
		\item $\hat{x}$处的最速下降方向为$d=-\bigtriangledown f(\hat{x})=[344,-56]^T$
	\end{enumerate}
\end{proof}

\section{}
设有函数$f(x)=\frac{1}{2}x^TAx+b^Tx+c$,其中$A$为对称正定矩阵。又设$x^{(1)}\ne \bar{x}$可表示为$x^{(1)}=\bar{x}+\mu p$,其中$\bar{x}$是$f(x)$的极小点,$p$是$A$的属于特征值$\lambda$的特征向量。证明:
\begin{equation}
\begin{aligned}
&(1)\quad \bigtriangledown f(x^{(1)})=\mu \lambda p \\
&(2)\quad \mbox{如果从}x^{(1)}\mbox{出发,沿最速下降方向做一维搜索,则一步达到极小点}\bar{x}
\end{aligned}
\end{equation}

\begin{proof}
	\begin{enumerate}[(1)]
		\item 计算$\bigtriangledown f(x)=Ax+b$,由于$\bar{x}$是$f(x)$的极小值点,因此$\bigtriangledown f(\bar{x})=A\bar{x}+b=0$,将$x^{(1)}=\bar{x}+\mu p$带入得:
		\begin{equation}
		\begin{aligned}
		\bigtriangledown f(x^{(1)})&=A(x^{(1)})+b \\
		&=A(\bar{x}+\mu p)+b \\
		&=A\mu p +A\bar{x}+b \\
		&=\mu A p=\mu \lambda p
		\end{aligned}
		\end{equation}
		
		\item 由(1)中的结论,从$x^{(1)}$出发进行一维搜索,$x^{(2)}=x^{(1)}-\beta\bigtriangledown f(x^{(1)})=\bar{x}+(1-\beta\lambda)\mu p$
		
		由于$A$对称正定,因此$\lambda>0$,因此取$\beta=1/\lambda$即可一步迭代达到极小点$\bar{x}$
	\end{enumerate}
\end{proof}

\section{}
设$A$为$n$阶实对称正定矩阵,证明$A$的$n$个互相正交的特征向量$p^{(1)},p^{(2)},\cdots,p^{(n)}$关于$A$共轭

\begin{proof}
	设$Ap^{i}=\lambda_ip^{(i)},i=1,2,\cdots,n$,已知$p^{(i)T}p^{(j)}=0,i\ne j$,因此对于任意$i\ne j,i,j=1,2,\cdots,n$都有:
	\begin{equation}
	p^{(i)T}Ap^{(j)}=p^{(i)T}\lambda_jp^{(j)}=\lambda_jp^{(i)T}p^{(j)}=0
	\end{equation}
	所以,$p^{(1)},p^{(2)},\cdots,p^{(n)}$关于$A$共轭
\end{proof}

\section{}
设$A$为对称正定矩阵,非零向量$p^{(1)},p^{(2)},\cdots,p^{(n)}\in E^n$关于矩阵$A$共轭。证明:
\begin{equation}
\begin{aligned}
&(1)\quad x=\sum_{i=1}^{n}\frac{p^{(i)T}Ax}{p^{(i)T}Ap^{(i)}}p^{(i)},\forall x\in E^n \\
&(2)\quad A^{-1}=\sum_{i=1}^{n}\frac{p^{(i)}p^{(i)T}}{p^{(i)T}Ap^{(i)}}
\end{aligned}
\end{equation}

\begin{proof}
	\begin{enumerate}[(1)]
		\item 由题设可知,非零向量$p^{(1)},p^{(2)},\cdots,p^{(n)}\in E^n$线性无关,所以可以构成$E^n$的一组基,因此可以将$x$表示为其线性组合的形式:
		\begin{equation}
		\label{eq1}
		x=\sum_{i=1}^{n}\lambda_ip^{(i)}
		\end{equation}
		对上式两边左乘$p^{(i)T}A$,又由于$p^{(i)T}Ap^{(j)}=0,\forall i\ne j$,因此:
		\begin{equation}
		\begin{aligned}
		p^{(i)T}Ax&=p^{(i)T}\sum_{i=1}^{n}\lambda_iAp^{(i)}\\
		&=\lambda_ip^{(i)T}Ap^{(i)}
		\end{aligned}
		\end{equation}
		所以:
		\begin{equation}
		\lambda_i=\frac{p^{(i)T}Ax}{p^{(i)T}Ap^{(i)}}
		\end{equation}
		带入\ref{eq1}中即可得到:
		\begin{equation}
		x=\sum_{i=1}^{n}\frac{p^{(i)T}Ax}{p^{(i)T}Ap^{(i)}}p^{(i)},\forall x\in E^n
		\end{equation}
		
		\item 不妨设$A^{-1}=[\beta_1,\beta_2,\cdots,\beta_n]$,则由第一问的结论:
		\begin{equation}
		\beta_j=\sum_{i=1}^{n}\frac{p^{(i)T}A\beta_j}{p^{(i)T}Ap^{(i)}}p^{(i)}=\sum_{i=1}^{n}\frac{p^{(i)}p^{(i)T}A\beta_j}{p^{(i)T}Ap^{(i)}},\forall j=1,2,\cdots,n
		\end{equation}
		所以:
		\begin{equation}
		\begin{aligned}
		A^{-1}&=[\beta_1,\beta_2,\cdots,\beta_n] \\
		&=\sum_{i=1}^{n}\frac{p^{(i)}p^{(i)T}A[\beta_1,\beta_2,\cdots,\beta_n]}{p^{(i)T}Ap^{(i)}} \\
		&=\sum_{i=1}^{n}\frac{p^{(i)}p^{(i)T}AA^{-1}}{p^{(i)T}Ap^{(i)}} \\
		&=\sum_{i=1}^{n}\frac{p^{(i)}p^{(i)T}I}{p^{(i)T}Ap^{(i)}} \\
		&=\sum_{i=1}^{n}\frac{p^{(i)}p^{(i)}T}{p^{(i)T}Ap^{(i)}} 
		\end{aligned}
		\end{equation}
		即:
		\begin{equation}
		A^{-1}=\sum_{i=1}^{n}\frac{p^{(i)}p^{(i)T}}{p^{(i)T}Ap^{(i)}}
		\end{equation}
	\end{enumerate}
\end{proof}

\section{}
设有非线性规划问题:
\begin{equation}
\begin{aligned}
\min \quad&\frac{1}{2}x^TAx \\
s.t.\quad &x\ge b
\end{aligned}
\end{equation}
其中$A$为n阶对称正定矩阵。设$\bar{x}$是问题的最优解,证明:$\bar{x}$与$\bar{x}-b$关于$A$共轭

\begin{proof}
	由于此优化问题是凸规划问题,KKT条件如下:
	\begin{subequations}
	\begin{numcases}{}
	A\bar{x}-w^T=0 \label{subeq1}\\
	w(\bar{x}-b)=0 \label{subeq2}\\
	w\ge 0
	\end{numcases}
	\end{subequations}
\end{proof}
由式\ref{subeq1}可得$w=\bar{x}^TA$,带入式\ref{subeq2}得,$\bar{x}^TA(\bar{x}-b)=0$,即$\bar{x}$与$\bar{x}-b$关于$A$共轭
\end{document}